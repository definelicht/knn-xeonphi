\section{Introduction}
  \label{sec:intro}

  % Do not start the introduction with the abstract or a slightly modified
  % version. It follows a possible structure of the introduction. 
  % Note that the structure can be modified, but the
  % content should be the same. Introduction and abstract should fill at most the first page, better less.

  \mypar{Motivation} In many applications, efficient search algorithms are of 
  great importance due to their high computational cost.  A common search 
  operation is to find a closest neighbor given some query point, which is 
  referred to as \emph{nearest neighbor search}.  This operation can readily be 
  extended to find the set of the $k$-nearest neighbors instead of just the 
  closest nearest neighbor.  The search space is usually very large and each 
  data point in such a set can be of high dimensionality.  For example, 
  matching images using distinct local image features~\cite{lowe2004a} is 
  a common task performed by web-search engines or in image recognition using 
  large datasets of up to $80$~million small images~\cite{torralba2008a}, where 
  each of the images is a high dimensional data point, depending on the number 
  of pixels used to represent the image.

  In order to speed-up the basic linear traversal of the data during the 
  search, accelerating data structures are employed.  For example, the k-d tree 
  is a simple such structure~\cite{bentley1975a,friedman1977a}.  However, exact 
  nearest neighbor search using a k-d tree is only efficient for low dimensional 
  data~\cite{muja2009a}.  For high dimensional data approximate structures, 
  such as a randomized k-d tree~\cite{silpa2008a} or hierarchical $k$-means 
  tree~\cite{muja2009a}, are used at the cost of the accuracy of the returned 
  nearest neighbors.

  In contrast to the naive linear search, an additional overhead is created by 
  building the accelerating data structure before performing the search.  
  Hence, the overall search time is composed of a build time for the data 
  structure and the time to perform the search using the structure.  Naturally, 
  the build time for a linear search vanishes.

  In this work, we focus on the fast parallel build of a set of randomized 
  k-d trees used to perform a parallel $k$-nearest neighbor search on high 
  dimensional data.  For our parallel application, we utilize the Intel many 
  integrated core (MIC) architecture offered by the Intel Knights Corner (KNC) 
  co-processor family.

  % \mypar{Motivation} The first task is to motivate what you do.  You can
  % start general and zoom in one the specific problem you consider.  In
  % the process you should have explained to the reader: what you are doing,
  % why you are doing, why it is important (order is usually reversed).

  % For example, if my result is the fastest sorting implementation ever, one
  % could roughly go as follows. First explain why sorting is important
  % (used everywhere with a few examples) and why performance matters (large datasets,
  % realtime). Then explain that fast implementations are very hard and
  % expensive to get (memory hierarchy, vector, parallel). 

  % Now you state what you do in this paper. In our example: 
  % presenting a sorting implementation that is
  % faster for some sizes as all the other ones.

  \mypar{Related work} Silpa-Anan and Hartley~\cite{silpa2008a} have proposed 
  a k-d tree algorithm based on creating multiple randomized k-d trees, where 
  each of the randomized trees splits the data based on a random choice of the 
  $D$ dimensions with highest variance.  Muja and Lowe have developed a C++ 
  library\footnote{\url{http://www.cs.ubc.ca/research/flann/}} for fast 
  approximate nearest neighbor searches with automatic algorithm selection 
  based on the data used for the search~\cite{muja2009a,muja2014a}.  The same 
  library implements a parallel build of a k-d tree for $3$D~data using GPU 
  hardware.  Building trees for high dimensional data is not parallelized.  The 
  work of Garcia~et~al.\@ employ GPUs to perform basic linear $k$-nearest 
  neighbor searches by using optimized linear algebra libraries and sorting 
  algorithms~\cite{garcia2008a,garcia2010a}.

  % \mypar{Related work} Next, you have to give a brief overview of
  % related work. For a report like this, anywhere between 2 and 8
  % references. Briefly explain what they do. In the end contrast to what
  % you do to make now precisely clear what your contribution is.

  % mainfile: ./../report.tex
