\section{Conclusions}
\label{sec:conclusions}

% Here you need to summarize what you did and why this is
% important. {\em Do not take the abstract} and put it in the past
% tense. Remember, now the reader has (hopefully) read the report, so it
% is a very different situation from the abstract. Try to highlight
% important results and say the things you really want to get across
% such as high-level statements (e.g., we believe that .... is the right
% approach to .... Even though we only considered x, the
% .... technique should be applicable ....) You can also formulate next
% steps if you want. Be brief. After the conclusions there are only the references.

The parallel build scheme for randomized k-d trees presented here has been shown to scale to a high number of cores, reducing the build time for four trees from one million SIFT descriptors to roughly half a second on a node consisting of 24 cores. This makes it feasible to construct and query randomized k-d trees in real-time applications even for very large data sets. Search performance on the Xeon Phi was shown to match that of a single 12-core Xeon E5-2697v2 socket
without explicit Xeon Phi optimizations. Although this corresponds to a significantly worse performance per dollar with the high price point of the Xeon Phi, it incites some optimism to the potential of the coprocessor, especially for the introduction of the Knight's Landing generation, which should significantly improve cache performance, as well as doubling the number of available cores. 

% mainfile: ./../report.tex
