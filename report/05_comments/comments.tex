\section{Further Comments}
  \label{sec:further_comments}

  Here we provide some further tips.

  \mypar{Further general guidelines}

  \begin{itemize}
    \item For short papers, to save space, I use paragraph titles instead of
      subsections, as shown in the introduction.

    \item It is generally a good idea to break sections into such smaller
      units for readability and since it helps you to (visually) structure the story.

    \item The above section titles should be adapted to more precisely
      reflect what you do.

    \item Each section should be started with a very
      short summary of what the reader can expect in this section. Nothing
      more awkward as when the story starts and one does not know what the
      direction is or the goal.

    \item Make sure you define every acronym you use, no matter how
      convinced you are the reader knows it.

    \item Always spell-check before you submit (to us in this case).

    \item Be picky. When writing a paper you should always strive for very
      high quality. Many people may read it and the quality makes a big difference.
      In this class, the quality is part of the grade.

    \item Books helping you to write better: \cite{Higham:98} and \cite{Strunk:00}.

    \item Conversion to pdf (latex users only): 

      dvips -o conference.ps -t letter -Ppdf -G0 conference.dvi

      and then

      ps2pdf conference.ps
    \end{itemize}

    \mypar{Graphics} For plots that are not images {\em never} generate the bitmap formats
    jpeg, gif, bmp, tif. Use eps, which means encapsulate postscript. It is
    scalable since it is a vector graphic description of your graph. E.g.,
    from Matlab, you can export to eps.

    The format pdf is also fine for plots (you need pdflatex then), but only if the plot was never before in the format 
    jpeg, gif, bmp, tif.



    % mainfile: ./../report.tex
